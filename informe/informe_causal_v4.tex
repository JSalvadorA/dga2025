\documentclass[11pt,a4paper]{article}
\usepackage[spanish]{babel}
\usepackage[utf8]{inputenc}
\usepackage[T1]{fontenc} 
\usepackage{lmodern}
\usepackage[a4paper,margin=2.5cm]{geometry}
\usepackage{setspace}
\usepackage{parskip}
\usepackage{enumitem}
\usepackage{hyperref}
\usepackage{booktabs}
\usepackage{tabularx}
\usepackage{siunitx}
\usepackage[table]{xcolor}
\usepackage{amsmath}
\usepackage{float}
\usepackage{graphicx}
\usepackage{tcolorbox}
\usepackage{caption}
\usepackage{needspace}

% Caja de analisis con fondo igual al resumen ejecutivo
\newtcolorbox{analisis}{
  colback=yellow!10,
  colframe=yellow!50!black,
  boxrule=0.5pt,
  arc=3pt,
  left=6pt,
  right=6pt,
  top=4pt,
  bottom=4pt,
  fontupper=\small
}

\hypersetup{colorlinks=true,linkcolor=black,urlcolor=black,hypertexnames=false}
\urlstyle{same}
\Urlmuskip=0mu plus 1mu
\def\UrlBreaks{\do\/\do-\do\_\do\.\do\:\do\?}
\setstretch{1.1}
\setlist{topsep=0.25em,itemsep=0.2em,leftmargin=1.2cm}
\setlength{\parskip}{0.65em}
\captionsetup[figure]{skip=2pt}
\setlength{\textfloatsep}{8pt}
\setlength{\intextsep}{6pt}
\newcommand{\placeholder}[1]{\textcolor{gray}{\textit{[#1]}}}
\newcommand{\figinclude}[1]{\includegraphics[width=\textwidth]{\detokenize{../outputs/figuras_finales/#1}}}

\begin{document}

\begin{center}
{\Large\textbf{Informe t\'ecnico: An\'alisis del salto del Indicador V4 (cumple\_v4), 2022--2025}}\\[0.6em]
{\normalsize\textbf{Automatizaci\'on de datos presupuestales y an\'alisis causal del salto 2025 (SIGA Web -- CMN)}}\\[0.2em]
{\small\textbf{Fecha: Enero 2026}}\\[0.3em]
\end{center}

\section*{Resumen Ejecutivo}

\begin{tcolorbox}[colback=yellow!10, colframe=yellow!50!black, boxrule=0.5pt, arc=3pt]
\textbf{Hallazgo central:} El \textbf{79\%} del salto 2024$\rightarrow$2025 en el Indicador V4
se explica porque las mismas entidades (ALWAYS\_IN) mejoraron su cumplimiento,
no por la entrada de nuevas entidades al padr\'on.
\end{tcolorbox}

\begin{itemize}[leftmargin=1cm, itemsep=0.3em]
  \item \textbf{Magnitud del salto:} De 33\% (2022--2024) a 74\% (2025), equivalente a +40 pp en un a\~no.
  \item \textbf{Descomposici\'on Oaxaca-Blinder:} 78.8\% comportamiento (ALWAYS\_IN) + 21.2\% composici\'on (ENTRY).
  \item \textbf{Efecto generalizado:} Todos los quintiles presupuestales (Q1--Q5) muestran saltos $>$65 pp.
  \item \textbf{Hip\'otesis:} La transici\'on SIGA Escritorio $\rightarrow$ SIGA Web simplific\'o el registro del CMN.
  \item \textbf{Limitaci\'on principal:} An\'alisis descriptivo; no hay grupo control puro para inferencia causal estricta.
\end{itemize}

\textbf{M\'etodos empleados:} Event Study (TWFE), DiD cl\'asico y con PSM, descomposici\'on Oaxaca-Blinder,
an\'alisis de heterogeneidad por tama\~no presupuestal, tests de placebo.

\vspace{0.5em}
\hrule
\vspace{0.5em}

\section*{I. Motivaci\'on}
En 2025 se observa un salto abrupto en el cumplimiento del Indicador 1, variante V4
(cumple\_v4): de $\sim$33\% (2022--2024) a 74\% (2025). Este indicador resume el
cumplimiento integral de las tres fases de la PMBSO (programaci\'on del CMN) en el
marco de la Directiva N\textsuperscript{o} 0007-2025-EF/54.01.

El contexto institucional sugiere que este cambio coincide con la transici\'on del
SIGA Escritorio hacia SIGA Web y con el proceso de implementaci\'on progresiva
establecido en la Segunda DCF del DL 1439.

\textbf{Objetivo:} Establecer una base trazable que permita distinguir qu\'e parte
del salto se explica por cambios en el comportamiento de las mismas entidades y
qu\'e parte se asocia a cambios de composici\'on del padr\'on, dejando expl\'icitas
las limitaciones de identificaci\'on.

El informe descompone el salto en:
\begin{itemize}
  \item cambio en el comportamiento de las mismas entidades (\textbf{ALWAYS\_IN}:
        UEs con SIGA=SI en 2022--2024 y en 2025),
  \item cambio por transici\'on de estado (\textbf{SWITCHER}: UEs con SIGA=NO en
        2022--2024 que pasan a SIGA=SI en 2025),
  \item cambio por composici\'on del padr\'on (\textbf{ENTRY}: UEs nuevas que solo
        aparecen en el padr\'on 2025),
  \item heterogeneidad por tama\~no presupuestal (PIA/PIM).
\end{itemize}

\textbf{Nota:} SWITCHER y ENTRY son grupos conceptualmente distintos (transici\'on de estado
vs entrada al padr\'on), aunque num\'ericamente casi iguales (607 vs 608 UEs) debido a
filtros de datos.

El objetivo es ofrecer evidencia trazable para el diagn\'ostico del salto 2025,
con un enfoque descriptivo y metodol\'ogicamente transparente.

\begin{table}[H]
\centering
\caption{Indicador 1 por variante, 2022--2025 (columna 2025 resaltada)}
\label{tab:ind1_v4_motivacion}
\begin{tabular*}{\textwidth}{@{\extracolsep{\fill}}lcccc}
\toprule
Variante & 2022 & 2023 & 2024 & \cellcolor{yellow!20}\textbf{2025} \\
\midrule
V1 -- Identificaci\'on & 39.46\% & 47.96\% & 43.90\% & \cellcolor{yellow!20}78.05\%\\
V2 -- Clasificaci\'on y Priorizaci\'on & 39.32\% & 46.61\% & 38.17\% & \cellcolor{yellow!20}75.52\%\\
V3 -- Consolidaci\'on y Aprobaci\'on & 33.71\% & 40.71\% & 35.23\% & \cellcolor{yellow!20}74.29\%\\
V4 -- Todas las fases (intersecci\'on) & 33.27\% & 40.34\% & 34.68\% & \cellcolor{yellow!20}\textbf{74.22\%}\\
\bottomrule
\end{tabular*}
\end{table}

\begin{figure}[H]
\centering
\includegraphics[width=0.95\textwidth]{\detokenize{../../results/latex_cuadros_figuras_ind1/figura1_evolucion_indicador_v4.png}}
\caption{Evoluci\'on del Indicador 1, variante V4 (2022--2025).}
\label{fig:v4_evolucion}
\end{figure}

\Needspace{10\baselineskip}
\subsection*{I.1 Modificaci\'on normativa (DL 1439)}
El marco legal establece dos hechos clave: (i) el registro es obligatorio para el sector p\'ublico,
``El registro de la informaci\'on relacionada con el Sistema Nacional de Abastecimiento es de uso
obligatorio por parte de las entidades del Sector P\'ublico'' (Art. 25); y (ii) la implementaci\'on
es gradual, ``La implementaci\'on del SIGA, establecido en el Subcap\'itulo V, es progresiva en las
entidades del Sector P\'ublico, de acuerdo a las directivas que emita la Direcci\'on General de
Abastecimiento'' (Segunda DCF). Esto sustenta que el cambio 2025 ocurre dentro de un mandato de
obligatoriedad con adopci\'on progresiva.

\section*{II. Pre-an\'alisis (datos y preparaci\'on)}
\textbf{Fuentes principales:}
\begin{itemize}
  \item Padr\'on de entidades con SIGA implementado (DPIP).
  \item Registros de CMN (SIGA MEF y SIGA MINEDU).
  \item Scraping de presupuestos (PIA, PIM, Devengado) para municipalidades,
        obtenido de \textbf{Consulta Amigable MEF} para los a\~nos \textbf{2022--2025}.
        En 2025, solo se scrapearon entidades con SIGA=SI.
  \item Directiva N\textsuperscript{o} 0007-2025-EF/54.01 (PMBSO/CMN).
\end{itemize}

\textbf{Construcci\'on de variables clave:}
\begin{itemize}
  \item \textbf{cumple\_v4}: indicador binario (1 si cumple las 3 fases CMN).
  \item \textbf{y\_exec\_pct}: porcentaje de ejecuci\'on presupuestal
        \texttt{devengado / pim * 100}; se usa como outcome continuo complementario.
  \item \textbf{Grupos seg\'un padr\'on y estado SIGA}:
    \begin{itemize}
      \item ALWAYS\_IN: SIGA=SI en 2022--2024 y SIGA=SI en 2025 (1,284 UEs).
      \item SWITCHER: SIGA=NO en 2022--2024, SIGA=SI en 2025 (607 UEs).
      \item ENTRY: ausentes en 2022--2024, presentes en padr\'on 2025 (608 UEs).
    \end{itemize}
  \item \textbf{Quintiles de PIA/PIM}: definidos con PIA/PIM 2024 (estables).
\end{itemize}

\clearpage
\section*{III. Resultados principales}
\textbf{Nota de alcance:} los modelos DiD/PSM se estiman solo con ALWAYS\_IN y SWITCHER
(panel T1); el grupo ENTRY se utiliza en la descomposici\'on Oaxaca-Blinder.

\Needspace{10\baselineskip}
\subsection*{III.1 Event Study (cumple\_v4)}
\textbf{Parte A (descriptivo, ALWAYS\_IN):} el salto 2025 es masivo y supera la
tendencia previa 2022--2024.

\begin{table}[H]
\centering
\caption{Cumple\_v4 por grupo (tasa)}
\begin{tabular*}{\textwidth}{@{\extracolsep{\fill}}lccc}
\toprule
Grupo & 2024 & 2025 & Salto\\
\midrule
ALWAYS\_IN & 0.2274 & 0.8886 & +66 pp\\
SWITCHER & 0.0049 & 0.6046 & +60 pp\\
\bottomrule
\end{tabular*}
\end{table}
\textbf{Nota:} tasas expresadas como proporciones (0--1), equivalentes a porcentajes.

\begin{figure}[H]
\centering
\includegraphics[width=0.8\textwidth]{\detokenize{../outputs/figuras_finales/fig1_coeff_partA_latex.png}}
\caption{Event Study (Parte A): coeficientes por año en ALWAYS\_IN.}
\label{fig:es_partA_coeff}
\end{figure}

\begin{analisis}
\textbf{Figura~\ref{fig:es_partA_coeff}:} Los coeficientes $\beta_{2023}=0.08$ y $\beta_{2024}=0.09$ muestran una tendencia
positiva leve en 2022--2024. El coeficiente $\beta_{2025}=0.75$ representa un salto abrupto de 75 pp,
8 veces mayor que la tendencia previa. Esto confirma que 2025 es un a\~no at\'ipico.
\end{analisis}

\begin{figure}[H]
\centering
\includegraphics[width=0.8\textwidth]{\detokenize{../outputs/figuras_finales/fig2_parallel_trends_latex.png}}
\caption{Event Study: tendencias por grupo (ALWAYS\_IN vs SWITCHER).}
\label{fig:es_parallel_trends}
\end{figure}

\begin{analisis}
\textbf{Figura~\ref{fig:es_parallel_trends}:} ALWAYS\_IN (l\'inea superior) parte de 14\% en 2022 y alcanza 89\% en 2025.
SWITCHER (l\'inea inferior) parte de casi 0\% y salta a 60\% en 2025. Ambos grupos experimentan
el salto 2025, pero ALWAYS\_IN mantiene tasas consistentemente m\'as altas en todo el per\'iodo.
\end{analisis}

\textbf{Parte B (contraste SWITCHER vs ALWAYS\_IN):} los SWITCHER saltan menos en
t\'erminos absolutos. El contraste es descriptivo, no causal, debido a
pre-trends no paralelos y problemas num\'ericos (SE NaN, $R^2$ negativo).
Los coeficientes se reportan solo como descriptivos (sin inferencia).

\begin{figure}[H]
\centering
\includegraphics[width=0.8\textwidth]{\detokenize{../outputs/figuras_finales/fig3_coeff_partB_latex.png}}
\caption{Event Study (Parte B): contraste SWITCHER vs ALWAYS\_IN.}
\label{fig:es_partB_coeff}
\end{figure}

\begin{analisis}
\textbf{Figura~\ref{fig:es_partB_coeff}:} Los coeficientes de contraste ($\beta_{switcher,t}$) son negativos en todos los a\~nos:
$-0.07$ (2023), $-0.06$ (2024), $-0.10$ (2025). Esto indica que SWITCHER est\'a sistem\'aticamente
por debajo de ALWAYS\_IN. El coeficiente 2025 m\'as negativo ($-0.10$) refleja que SWITCHER
salta 10 pp MENOS que ALWAYS\_IN en t\'erminos relativos.
\end{analisis}

\begin{figure}[H]
\centering
\includegraphics[width=0.85\textwidth]{\detokenize{../outputs/figuras_finales/fig10_transition_latex.png}}
\caption{Transici\'on 2024$\rightarrow$2025 en cumple\_v4 (salto de entidades).}
\label{fig:transition_2024_2025}
\end{figure}

\begin{analisis}
\textbf{Figura~\ref{fig:transition_2024_2025}:} Matriz de transici\'on: 1,104 entidades pasaron de no\_cumple a cumple (cuadrante
inferior-izquierdo), explicando el salto masivo. Solo 150 entidades retrocedieron (cumple$\rightarrow$no\_cumple).
El flujo neto positivo de 954 entidades (+1,104$-$150) representa el grueso del incremento 2025.
\end{analisis}

\Needspace{10\baselineskip}
\subsection*{III.2 Oaxaca-Blinder}
\textbf{Descomposici\'on agregada (2024 $\rightarrow$ 2025):}
\begin{table}[H]
\centering
\caption{Oaxaca-Blinder agregado}
\begin{tabular*}{\textwidth}{@{\extracolsep{\fill}}lcc}
\toprule
Componente & Valor (pp) & Share\\
\midrule
Delta total & 56.96 & 100\%\\
Comportamiento (ALWAYS\_IN) & 44.87 & 78.8\%\\
Composici\'on (ENTRY) & 12.09 & 21.2\%\\
\bottomrule
\end{tabular*}
\end{table}

\begin{figure}[H]
\centering
\figinclude{fig4_waterfall_multiyear_latex.png}
\caption{Oaxaca-Blinder: descomposici\'on multianual (comportamiento vs composici\'on).}
\label{fig:oaxaca_waterfall}
\end{figure}

\begin{analisis}
\textbf{Figura~\ref{fig:oaxaca_waterfall}:} Gr\'afico waterfall mostrando la descomposici\'on del salto 2024$\rightarrow$2025:
el componente ``comportamiento'' (barra azul) aporta 44.9 pp, mientras que ``composici\'on''
(barra naranja) aporta 12.1 pp. El salto total de 57 pp est\'a dominado por cambios dentro
de las mismas entidades, no por entrada de nuevas.
\end{analisis}

\begin{figure}[H]
\centering
\figinclude{fig5_share_2025_latex.png}
\caption{Oaxaca-Blinder 2024$\rightarrow$2025: participaci\'on del salto por componente.}
\label{fig:oaxaca_share}
\end{figure}

\begin{analisis}
\textbf{Figura~\ref{fig:oaxaca_share}:} Gr\'afico de torta: 78.8\% del salto corresponde a ``comportamiento''
(ALWAYS\_IN mejor\'o su tasa de 22.7\% a 88.9\%) y 21.2\% a ``composici\'on'' (entrada de
608 nuevas entidades con tasa 60.4\%). Conclusi\'on: el driver principal es el cambio
interno, no la recomposici\'on del padr\'on.
\end{analisis}

\begin{figure}[H]
\centering
\figinclude{fig6_rates_evolution_latex.png}
\caption{Evoluci\'on de tasas por grupo (ALWAYS\_IN y ENTRY).}
\label{fig:oaxaca_rates}
\end{figure}

\begin{analisis}
\textbf{Figura~\ref{fig:oaxaca_rates}:} Evoluci\'on temporal de tasas: ALWAYS\_IN (l\'inea continua) muestra crecimiento
gradual 2022--2024 y salto abrupto en 2025. ENTRY (marcador 2025) aparece con tasa 60.4\%,
inferior a ALWAYS\_IN (88.9\%) pero superior a la tasa base 2024 (22.7\%). Las nuevas entidades
llegan con nivel de cumplimiento intermedio.
\end{analisis}

\textbf{Interpretaci\'on:} el salto se explica principalmente por cambio dentro
de las mismas entidades (ALWAYS\_IN), no por la entrada de nuevas entidades.
La descomposici\'on individual con PIA/PIM tiene $R^2$ muy bajo ($<$2\%), lo que indica
que el salto se asocia a factores no observados (coherente con el cambio de
plataforma en 2025).

\textbf{Nota:} ENTRY (608 UEs) y SWITCHER (607 UEs) son grupos distintos
(composici\'on vs transici\'on SIGA), aunque num\'ericamente casi iguales.

\Needspace{10\baselineskip}
\subsection*{III.3 Heterogeneidad por tama\~no (PIA/PIM)}
\textbf{Hallazgo:} el salto es generalizado y la heterogeneidad es moderada.

\begin{table}[H]
\centering
\caption{Efecto post\_2025 por quintil PIA}
\begin{tabular*}{\textwidth}{@{\extracolsep{\fill}}lccc}
\toprule
Quintil & $\beta_{\text{post\_2025}}$ & SE & $\Delta$ vs Q1\\
\midrule
Q1 (peque\~nas) & 0.6524 & 0.0183 & ---\\
Q2 & 0.6783 & 0.0438 & +2.6 pp\\
Q3 & 0.6782 & 0.0300 & +2.6 pp\\
Q4 & 0.7432 & 0.0356 & +9.1 pp\\
Q5 (grandes) & 0.7198 & 0.0371 & +6.7 pp\\
\bottomrule
\end{tabular*}
\end{table}

\textbf{Lectura:} Q1 (entidades peque\~nas) salta 65.2 pp. Q4 salta 74.3 pp (+9.1 pp m\'as que Q1).

\begin{figure}[H]
\centering
\figinclude{fig7_quintile_effects_latex.png}
\caption{Heterogeneidad por quintil PIA: efectos post\_2025.}
\label{fig:quintile_effects}
\end{figure}

\begin{analisis}
\textbf{Figura~\ref{fig:quintile_effects}:} Efectos $\beta_{post\_2025}$ por quintil: Q1=0.65, Q2=0.68, Q3=0.68, Q4=0.74, Q5=0.72.
Todos los quintiles muestran saltos superiores a 65 pp. El patr\'on NO es estrictamente mon\'otono:
Q4 (medianas-grandes) tiene el mayor efecto, superando incluso a Q5 (las m\'as grandes).
Heterogeneidad moderada; el salto es generalizado.
\end{analisis}

\begin{figure}[H]
\centering
\figinclude{fig8_interactions_latex.png}
\caption{Interacciones post\_2025 $\times$ quintil (base Q1).}
\label{fig:quintile_interactions}
\end{figure}

\begin{analisis}
\textbf{Figura~\ref{fig:quintile_interactions}:} Coeficientes de interacci\'on (diferencial vs Q1): Q2 y Q3 no son estad\'isticamente
distintos de Q1 (IC cruza cero). Q4 muestra +9.1 pp (significativo, $p<0.05$) y Q5 muestra +6.7 pp
(marginalmente significativo). Las entidades medianas-grandes (Q4) se benefician ligeramente m\'as
del cambio de plataforma.
\end{analisis}

\begin{figure}[H]
\centering
\figinclude{fig9_before_after_latex.png}
\caption{Tasas antes vs despu\'es (2024 vs 2025) por quintil.}
\label{fig:quintile_before_after}
\end{figure}

\begin{analisis}
\textbf{Figura~\ref{fig:quintile_before_after}:} Comparaci\'on antes/despu\'es: todas las barras 2025 (derecha) superan ampliamente
a las barras 2024 (izquierda) en todos los quintiles. La brecha 2024$\rightarrow$2025 es similar
en magnitud (60--74 pp) para todos los tama\~nos, confirmando que el salto es un fen\'omeno
generalizado, no focalizado en un segmento espec\'ifico.
\end{analisis}

\Needspace{10\baselineskip}
\subsection*{III.4 Diagn\'osticos adicionales}

\textbf{F-test de tendencia previa en ALWAYS\_IN (Event Study Parte A):}
\begin{itemize}
  \item $H_0$: $\beta_{2023} = \beta_{2024} = 0$ (sin tendencia previa)
  \item Resultado: Wald = 130.68, $p < 0.001$ $\Rightarrow$ Se rechaza $H_0$.
  \item Interpretaci\'on: existe tendencia positiva 2022--2024, pero el salto 2025
        ($\beta = 0.75$) es de magnitud mucho mayor que la tendencia previa ($\beta \approx 0.08$--$0.09$).
\end{itemize}

\textbf{Bootstrap Oaxaca-Blinder (500 repeticiones):}
\begin{table}[H]
\centering
\caption{Intervalos de confianza 95\% (Oaxaca)}
\begin{tabular*}{\textwidth}{@{\extracolsep{\fill}}lcc}
\toprule
Componente & Estimaci\'on (pp) & IC 95\%\\
\midrule
Delta total & 56.96 & [54.7, 59.3]\\
Comportamiento & 44.87 & [43.3, 46.9]\\
Composici\'on & 12.09 & [10.8, 13.3]\\
\bottomrule
\end{tabular*}
\end{table}

Ambos componentes tienen IC que no incluyen cero, confirmando la robustez de la descomposici\'on.

\Needspace{10\baselineskip}
\subsection*{III.5 Resumen visual}
\begin{figure}[H]
\centering
\figinclude{dashboard_resumen_latex.png}
\caption{Panel resumen de resultados principales.}
\label{fig:dashboard_resumen}
\end{figure}

\begin{analisis}
\textbf{Figura~\ref{fig:dashboard_resumen} (Dashboard):} S\'intesis visual de los tres componentes del an\'alisis:
(1) Event Study confirma salto abrupto 2025 ($\beta=0.75$) muy superior a tendencia previa ($\beta\approx0.08$);
(2) Oaxaca-Blinder atribuye 78.8\% del salto a cambio de comportamiento (ALWAYS\_IN) y 21.2\% a composici\'on;
(3) Heterogeneidad muestra efecto generalizado en todos los quintiles con diferencial moderado
(Q4 salta 9 pp m\'as que Q1). Conclusi\'on integrada: el salto 2025 es un hecho robusto,
dominado por cambios internos y no por recomposici\'on del padr\'on.
\end{analisis}

\Needspace{10\baselineskip}
\subsection*{III.6 Modelos complementarios (DiD y outcome continuo)}
\textbf{Prop\'osito:} corroborar el patr\'on del salto 2025 con
especificaciones alternativas, manteniendo el car\'acter descriptivo
del an\'alisis (no causal estricto).

\begin{table}[H]
\centering
\caption{Modelos complementarios: magnitud y direcci\'on del efecto}
\label{tab:did_complementarios}
\begin{tabular*}{\textwidth}{@{\extracolsep{\fill}}lccc}
\toprule
Modelo & Outcome & $\delta$ (pp) & SE \\
\midrule
DiD cl\'asico 2x2 & cumple\_v4 & -9.2 & 2.3 \\
PSM--DiD (ATT) & cumple\_v4 & -6.6 & 2.6 \\
DiD FE (outcome continuo) & y\_exec\_pct & -0.29 & 0.51 \\
\bottomrule
\end{tabular*}
\end{table}

\begin{figure}[H]
\centering
\includegraphics[width=0.95\textwidth]{\detokenize{../did_clasico_2x2/outputs/fig_did_lines.png}}
\caption{DiD cl\'asico: SWITCHER vs ALWAYS\_IN con contrafactual.}
\label{fig:did_lines}
\end{figure}

\begin{analisis}
\textbf{Figura~\ref{fig:did_lines}:} El gr\'afico muestra la trayectoria observada de ambos grupos
y el contrafactual (l\'inea punteada): si SWITCHER hubiera seguido la misma tendencia
que ALWAYS\_IN. La brecha roja representa el efecto DiD ($\delta = -9.2$ pp):
SWITCHER salta menos de lo que hubiera saltado bajo tendencias paralelas.
\end{analisis}

\begin{figure}[H]
\centering
\IfFileExists{../did_psm/outputs/fig_psm_balance.png}{
  \includegraphics[width=0.85\textwidth]{\detokenize{../did_psm/outputs/fig_psm_balance.png}}
}{
  \placeholder{Figura PSM balance no disponible en outputs.}
}
\caption{Balance de covariables antes y despu\'es del matching (PSM).}
\label{fig:psm_balance}
\end{figure}

\begin{analisis}
\textbf{Figura~\ref{fig:psm_balance}:} El matching por propensity score reduce el desbalance en
log(PIA) y log(PIM) de $|SMD| \approx 0.6$ a $|SMD| < 0.02$ (reducci\'on $>$97\%).
Esto permite comparar SWITCHER con ALWAYS\_IN de tama\~no presupuestal similar.
Aun as\'i, la brecha DiD persiste ($\delta = -6.6$ pp), sugiriendo que factores
no observables (capacidad t\'ecnica, experiencia previa) explican la diferencia.
\end{analisis}

\begin{figure}[H]
\centering
\IfFileExists{../did_outcome_continuo/outputs/fig_compare_outcomes.png}{
  \includegraphics[width=0.85\textwidth]{\detokenize{../did_outcome_continuo/outputs/fig_compare_outcomes.png}}
}{
  \placeholder{Figura comparaci\'on de outcomes no disponible en outputs.}
}
\caption{Comparaci\'on de efectos DiD: cumple\_v4 vs ejecuci\'on presupuestal.}
\label{fig:compare_outcomes}
\end{figure}

\begin{analisis}
\textbf{Figura~\ref{fig:compare_outcomes} (Lectura integrada):} (i) el DiD cl\'asico indica que SWITCHER
salta \textbf{$\sim$9 pp menos} que ALWAYS\_IN; (ii) al emparejar por PIA/PIM
(PSM--DiD), la brecha se reduce a \textbf{$\sim$7 pp}, sugiriendo que parte
de la diferencia se explica por observables; (iii) con outcome continuo
(ejecuci\'on presupuestal), el efecto diferencial es \textbf{cercano a cero}
y no estad\'isticamente significativo. En conjunto, los modelos
refuerzan la narrativa del informe: \textbf{salto masivo en 2025}, con
una brecha moderada entre SWITCHER y ALWAYS\_IN, y sin evidencia de
una ca\'ida en desempe\~no presupuestal asociada al cambio.
\end{analisis}

\Needspace{10\baselineskip}
\subsection*{III.7 Tests de Placebo (validaci\'on DiD)}
\textbf{Prop\'osito:} verificar la credibilidad del DiD mediante pruebas de falsificaci\'on
(Cunningham, 2021, cap. 9.5). Se reestima el DiD en escenarios donde \textbf{no deber\'ia
haber efecto}; si aparece ``efecto'', sugiere problemas de identificaci\'on.

\textbf{A) Placebo temporal:} fingir tratamiento en 2023 o 2024 (usando solo datos pre-2025).
\begin{table}[H]
\centering
\caption{Tests de placebo temporal}
\label{tab:placebo_temporal}
\begin{tabular*}{\textwidth}{@{\extracolsep{\fill}}lcccc}
\toprule
Test & $\delta$ & SE & $p$-value & Significativo \\
\midrule
Placebo 2023 & $-0.077$ & 0.015 & $<0.001$ & S\'i $\dagger$ \\
Placebo 2024 & $-0.045$ & 0.014 & 0.002 & S\'i $\dagger$ \\
\textbf{REAL 2025} & $-0.092$ & 0.023 & $<0.001$ & \textbf{S\'i} \\
\bottomrule
\end{tabular*}
\end{table}
$\dagger$ Advertencia: placebos significativos indican pre-trends no paralelos.

\textbf{Interpretaci\'on:} Los placebos temporales son significativos, lo que confirma
que SWITCHER ya estaba por debajo de ALWAYS\_IN antes de 2025. Sin embargo, el efecto
2025 ($-9.2$ pp) es \textbf{mayor} que los placebos ($-7.7$ y $-4.5$ pp), sugiriendo
que algo adicional ocurri\'o en 2025, posiblemente la transici\'on SIGA.

\textbf{B) Placebo outcome:} usar y\_exec\_pct (ejecuci\'on presupuestal), que no
deber\'ia reaccionar al tratamiento.
\begin{itemize}
  \item $\delta = -0.14$ (SE: 0.73), \textbf{NO significativo}.
  \item Buena se\~nal: no hay degradaci\'on de performance presupuestal.
  \item El efecto es espec\'ifico a cumple\_v4, no a eficiencia general.
\end{itemize}
\textbf{Nota metodol\'ogica:} este placebo usa FE de entidad (sin FE de tiempo),
por eso su magnitud no es directamente comparable con el DiD FE de la Tabla
\ref{tab:did_complementarios} (que incluye FE de tiempo).

\begin{analisis}
\textbf{Conclusi\'on de placebos:} (1) Pre-trends NO paralelos para cumple\_v4 (placebos
temporales significativos); (2) Efecto 2025 mayor que placebos previos; (3) Sin efecto
en ejecuci\'on presupuestal. El DiD debe interpretarse como \textbf{descriptivo}, no
como efecto causal puro. La brecha entre grupos exist\'ia antes de 2025, aunque se
ampli\'o con la transici\'on.
\end{analisis}

\section*{IV. Limitaciones}
\begin{itemize}
  \item El Event Study es descriptivo; no hay grupo control puro para ALWAYS\_IN.
  \item La descomposici\'on individual en Oaxaca usa solo PIA/PIM y tiene $R^2 < 2\%$,
        indicando que estas variables NO predicen cumple\_v4. El ``efecto coeficientes''
        captura factores no observados, no comportamiento literal.
  \item La heterogeneidad por tama\~no es exploratoria (no causal estricta).
  \item Los modelos DiD y PSM--DiD son descriptivos: no corrigen selecci\'on en no observables.
  \item El outcome continuo (ejecuci\'on presupuestal) no mide directamente la adopci\'on de SIGA;
        es un proxy complementario.
  \item Regresi\'on Discontinua (RD) fue evaluada y descartada: no existe running variable
        continua con cutoff significativo (ver viabilidad\_RD.md).
\end{itemize}

\section*{VI. Conclusiones}

\begin{enumerate}[leftmargin=1cm]
  \item \textbf{Salto robusto:} El incremento de $\sim$33\% a 74\% en cumple\_v4 (2025)
        es un hecho descriptivo s\'olido, confirmado por m\'ultiples m\'etodos.

  \item \textbf{Driver principal:} El 79\% del salto se explica por cambio de
        comportamiento de las mismas entidades (ALWAYS\_IN), no por entrada de nuevas.

  \item \textbf{Efecto generalizado:} Todas las entidades---peque\~nas y grandes---saltan
        $>$65 pp; heterogeneidad moderada (Q4 salta 9 pp m\'as que Q1).

  \item \textbf{Sin degradaci\'on operativa:} El placebo sobre ejecuci\'on presupuestal
        es no significativo, sugiriendo que el cambio no afect\'o el desempe\~no.

  \item \textbf{Limitaci\'on:} Los DiD deben interpretarse como descriptivos (pre-trends
        no paralelos). No se puede afirmar causalidad estricta.
\end{enumerate}

\textbf{Hip\'otesis sustantiva:} La transici\'on SIGA Escritorio $\rightarrow$ SIGA Web
parece haber simplificado el proceso de registro del CMN, facilitando el cumplimiento
formal de las tres fases.

\section*{Referencias}
\begin{itemize}
  \item Cunningham, S. (2021). \textit{Causal Inference: The Mixtape}, cap. 9.5.
  \item Roth, J. (2022). ``Pretrends in DiD: What to do when parallel trends fail''.
\end{itemize}

\end{document}
